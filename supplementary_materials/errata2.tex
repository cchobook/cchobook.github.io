\documentclass{errata}

\hypersetup{
    pdfauthor={王畅、林肃浩},
    pdftitle={中国化学奥林匹克竞赛初赛讲义·勘误表},
    pdfkeywords={}
}

\addbibresource{references.bib}

\title{中国化学奥林匹克竞赛初赛讲义 \\ {\bfseries 勘误表} \\ (第二次印刷)}
\author{王畅 \and 林肃浩}
\date{2024-06-08 \\ 更新于 \today}

\begin{document}
    \maketitle
    本文档的最新版本可访问 \url{https://cchobook.github.io/supplementary_materials/errata2.pdf} 下载。

    以下页码等信息参照\emph{浙江大学出版社 2024 年 6 月第二次印刷}之《中国化学奥林匹克竞赛初赛讲义》,ISBN 为 978-7-308-23901-1。条目结尾为提供反馈的读者署名,若无署名则为作者自行订正。
    
    旧版勘误:第一次印刷(\url{https://cchobook.github.io/supplementary_materials/errata1.pdf})

    \begin{Errata}
        \item[第 3 页,例题 1.6 解答第 2 个方程式]
            \Orig \ce{CH2O + MnO2 + 4H+ -> CO2 + 2Mn^{2+} + 3H2O}
            \Corr \ce{CH2O + 2MnO2 + 4H+ -> CO2 + 2Mn^{2+} + 3H2O}
            \Thx{匿名(利川市第五中学)}
        \item[第 302 页,注记]
            \Orig 环丙酮和环己酮的羰基吸收前者波数小……
            \Corr 环丙酮和环己酮的羰基吸收前者波数大,这是因为环丙酮环内张力大,羰基碳于环内 C-C 键的 p 成分增加,C=O 键的 s 成分增加,从而增强了 C=O 键。
            \Thx{匿名}
    \end{Errata}

    \renewcommand{\em}{\itshape}
    \renewcommand*{\bibfont}{\footnotesize}
    \renewcommand{\refname}{参考文献}
    \renewcommand{\bibname}{参考文献}
    \printbibliography
\end{document}
